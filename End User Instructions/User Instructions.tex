\documentclass[12pt,a4paper]{article}
\usepackage[utf8]{inputenc}
\usepackage[T1]{fontenc}
\usepackage{amsmath}
\usepackage[margin=3cm]{geometry}
\usepackage{amsfonts}
\usepackage{listings}
\usepackage{multicol}
\usepackage{amssymb}
\usepackage{graphicx}
\usepackage{color}
\usepackage{todonotes}
\lstset{
	basicstyle=\ttfamily,
	language=C++,
	backgroundcolor=\color[gray]{0.9},
	%% numbers=left, numberstyle=\footnotesize, stepnumber=5, 
	%% numbersep = 5pt, numberfirstline=false, firstnumber=last
}
\renewcommand\labelitemi{--}

\author{\textit{It prices options!} 
		\\ Luke Armitage}
\date{}
\title{{\bfseries End-User~Instructions}~for Two-Dimensional~Black-Scholes~Model 
Option~Pricer}


\begin{document}

\maketitle

\section{Introduction}
This program is designed to the specification provided as part of the project 
brief. It is designed to compute prices of American and European style stock 
options with a variety of payoff functions. It uses a two-dimensional 
Black-Scholes model, calibrated with historical stock data, and a discrete-time 
approximation of a continuous-time, two dimensional binary tree. It can output, 
to a spreadsheet, stock prices with strike prices in a given range, for a range 
of given maturity dates. 




\section{Running the software}
Unzip the folder \texttt{`Luke Armitage 109031954 -- code and executable.zip'} 
into the desired directory, and find the executable file 
\texttt{black-scholes-pricer.exe} in the directory \texttt{`\textbackslash 
bin\textbackslash Release'}.


\subsection{Calibrating the model}
Before it can be used to price options, the model must be calibrated using 
historical stock data. 
To calibrate the model, replace the file \texttt{data.csv}, found in the main 
directory, with your historical stock prices in \texttt{.csv} (comma delimited) 
file format. Upon running the program, you will be prompted to specify a time 
interval, which is the period between recorded historical stock prices. 

The error message ``\texttt{Error opening file. Closing program.}'' means that 
the program could not find the file \texttt{data.csv}. Please ensure the 
historical stock data is in the correct file format and saved in the correct 
directory.

After entering a time interval, the program will calculate the values from 
calibration from the given data. The program will confirm success and display 
the values in the following format:
\begin{lstlisting}
Calibration values:
   Historical Volatility of S0:  <value>
                            S1:  <value>
   Historical Correlation:       <value>
\end{lstlisting}
The first two values are the variability of each of the two stocks, and the 
third value is the correlation between the logarithmic returns of the two 
stocks. This gives a measure of how closely related the stocks' prices are: the 
sign (positive or negative) of the correlation gives the direction of the 
relationship, and the absolute value gives how strong the relationship is.

Next, the program will prompt you for the interest rate and current values of 
each stock, and ensure that the values are valid. If there is an error with the 
model, the program will give an error message and quit.


\subsection{Option Pricing}
Once the models have been calibrated and set up, the program can calculate 
option prices. The program will prompt for the expiry date of the option, and a 
payoff function. The default choices for payoff function are \textit{Spread 
call}, \textit{Min call} and \textit{Max call}. More choices can be added -- 
consult the Developer Documentation.

Once a payoff function has been selected, the program ask for the lower and 
upper strike prices $K_{min}$ and $K_{max}$ to evaluate between, and then set 
about calculating option prices.

The program will evaluate the option prices in the European option style and 
then the American option style, with eleven incremental strike prices from 
$K_{min}$ to $K_{max}$, with each of the 5 expiry dates from $100$ to $500$. As 
the program is calculating, dotted lines will indicate progress. Once each type 
of option has been  priced accordingly, the program will indicate that the 
prices can be found in the files \texttt{Eprices.csv} and \texttt{Aprices.csv} 
respectively, which can be found in the main directory for the program.
\end{document}
